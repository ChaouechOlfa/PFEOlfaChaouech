%% @Author: OLFA CHAOUECH
%  @Date:  09/2022
%% @Class:  PFE de l'ISET Rades.


\markboth{\MakeUppercase{Introduction}}{}%
\addcontentsline{toc}{chapter}{Introduction}%


Les services web ont envahi tous les secteurs d'activité tels que les secteurs du transport, de la santé, de l'économie et notamment le secteur de l'éducation.

Le recours à Internet pour gérer des évènements en ligne, des workshops, des formations, des réunions.., connaît une croissance rapide depuis de nombreuses années et particulièrement après la pandémie de COVID 2019.\\
En Tunisie, la plupart des institutions publiques et privées emploient les services gratuits offerts par les plateformes étrangères telles que "Microsoft Forms" et "Google Forms" pour gérer leurs évènements en ligne.\\
Cependant, l'instance nationale de protection des données personnelles (INPDP) a récemment déclaré que l'usage de ce type de services gratuits se traduit par une fuite des données massive dans le milieu universitaire.
Par conséquent, elle a rappelé dans son article publié le 28 octobre 2021, l'interdiction de recours à ce type de services.\\
Dans ce contexte, le Centre National Universitaire de Documentation Scientifique et Technique (CNUDST), souhaité automatiser la gestion des formations qu'il organise périodiquement au profit de la communauté scientifique tunisienne en développant sa propre application au lieu d'utiliser les services gratuits tel que "Google Forms".
Ce projet d'automatisation a été proposé en tant que PFE.\\
Ainsi, dans le cadre des études de troisième année en licence technologie de l'informatique spécialité "Développement des systèmes d'information (DSI)", nous avons effectué notre stage de Projet de Fin d'Etudes (PFE) au sein du CNUDST.\\
Ce dernier dispose d'un site web déployé moyennant le CMS wordpress qui utilise PHP comme langage de programmation et MySql comme base de données. Les modules de gestion des formations à développer intégreront le site web via le développement d'un nouveau thème pour assurer les inscriptions aux formations en ligne et un plugin pour gérer les formations et leurs inscriptions à ces formations.\\
Le présent rapport décrit la démarche que nous avons appliquée pour réaliser notre projet en utilisant l'approche SCRUM.\\
Notre travail est subdivisé en cinq chapitres comme suit: 
\begin{itemize}
\item Dans le premier chapitre «Cadre Général », nous allons présenter le
cadre du projet. En premier lieu, nous allons décrire l'entreprise d'accueil. En second lieu, nous allons mettre le focus sur l'analyse de l'existant et la solution proposée. Enfin, nous allons terminer par la précision du choix du framework de gestion du projet. 
\item Le second chapitre concerne la phase de planification de projet ou sprint 0. Cette phase comprend la spécification des différents besoins fonctionnels et non fonctionnels et le pilotage du projet avec la méthode SCRUM.\\
Nous allons présenter aussi notre environnement de travail matériel et logiciel ainsi que l'architecture logique et physique de notre système.
\item Les trois autres chapitres constituent le corps de notre rapport, ils seront consacrés au développement des sprints. Chaque chapitre fait l'objet d'un sprint.\\
Chaque sprint est un incrément du produit final qui est potentiellement livrable. 

\end{itemize}


